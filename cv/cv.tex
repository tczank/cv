\documentclass[a4paper]{article}
 
\usepackage{tabularx}
\usepackage [brazil, english, japanese]{babel}     % nomes e hifenaçã em português
\usepackage{CJKutf8}
\usepackage{graphicx} % Include figure files
\usepackage{t1enc}              % Permite digitar os acentos de forma normal
\usepackage[utf8]{inputenc}
\usepackage{ae}
\usepackage[T1]{fontenc}
\usepackage{CV}

\usepackage{hyperref}

\newcommand{\til}{\~{}}
\begin{document}

\pagestyle{empty}

%Ueberschrift
\begin{center}
\huge{\textsc{Curriculum Vitae}}
\vspace{\baselineskip}

\Large{\textsc{Czank Thomas Rafael}}
\end{center}
\vspace{1.5\baselineskip}

\section{Address}
\begin{CJK}{UTF8}{ipxm}
  \begin{flushleft}
〒252-0141  \\
神奈川県相模原市緑区相原2ー8ー20 \\
ファミーユ相原102号室 \\
E-mail: thomas.czank@gmail.com\\
\end{flushleft}
\end{CJK}

\section{Personal Info}
\begin{minipage}{0.5\textwidth}
\begin{flushleft}
  Birth Date: 1989/12/29  \\
  Place of Birth: São Paulo-SP, Brazil \\
  Nationality: Brazilian \\
  Marital Status: Married\\
  Cellphone number: +81-080-3339-8149\\
  Github: \url{https://github.com/tczank}
\end{flushleft}
\end{minipage}
\hfill
\begin{minipage}{0.5\textwidth}
\begin{flushright}
\hspace*{-2cm}
\includegraphics[scale=2.]{/home/tczank/Dropbox/job_app/niigatadai/rireki/3x4_3.png}
\end{flushright}
\end{minipage}

\section{Education}
\begin{CV}
\item[2008/03--2012/12] Bachelor degree in Physics - University of São Paulo
\item[2013/04--2015/08] Master's degree in Physics - Tohoku University \\
  \hspace*{-2cm}
  (\href{https://flex.phys.tohoku.ac.jp/japanese/riron/ronbun/m15thom.pdf}{Tip enhanced Raman spectroscopy of carbon nanotubes})
\item[2015/09--2018/11] PhD degree in Physics - Tohoku University \\
  \hspace*{-3cm} (\href{http://hdl.handle.net/10097/00125071}{The search for a dark sector muonic gauge boson $Z^'$ at the Belle experiment})
\end{CV}



\section{Languages}
\begin{itemize}
	\item English - Fluent
	\item Spanish - Fluent
  \item Japanese (N2) - Advanced
  \item Portuguese - Native Speaker
\end{itemize}

%\section{Participação em Eventos}
%\begin{itemize}
%	\item Siicusp 2010-Simpósio Internacional de Iniciação Científica da Universidade de São Paulo
%\end{itemize}

\section{Professional Experience}

\begin{CV}

% \item[2007/02--2018/11] \textbf{Occupation:} Private Tutor
% \\ \textbf{Activities:} Tutoring on English, Math and Physics to High School
% students preparing for University entrance examinations and also to University students.

%\begin{itemize}
%	\item \textbf{Preto} sempre;
%	\item \textbf{Colorido} quase nunca.
%\end{itemize}


% \item[2011/02--2012/03] \textbf{Ocupação:} Aluno de Iniciação Científica
% \\ \textbf{Instituição:} Laboratório de Instrumentação de Partículas no Departamento de Fìsica Geral no Instituto de Física da Universidade de São Paulo
% \\ \textbf{Professor:}Dr Marco Aurélio Lisboa Leite
% \\ \textbf{Atividades:} Acompanhamento e análise de eventos de calibração de Íons-Pesados do ATLAS e do ZDC(Zero Degree Calorimeter, Calorímetro de Zero Grau) detectores do Cern. E realizando simulações com o Fluka, um programa de física de altas energias com o método de Monte-Carlo, utilizando alvos de diferentes materiais e geometrias, e feixes de diferentes energias e partículas.
\item[2021/12--] \textbf{Occupation:} Postdoctoral Researcher
  \\ \textbf{Institution:} Tokyo Metropolitan University
  \\ \textbf{Supervisor:} \href{https://www-hep.phys.se.tmu.ac.jp/index.html}{Kakuno Hidekazu}
  \\ \textbf{Activities:} Statistical analysis for dark matter candidate detection using the data taken
  by the \href{https://belle.kek.jp}{Belle Experiment} during its 10 years of
  operation.
  The Belle Analysis Framework \texttt{basf} is based in \texttt{C/C++} it is
  used to select the particles and its variables relevant for the analysis.
  These variables relationship is studied to design an optimal
  set of selection criteria that will maximize the signal (dark matter
  candidate) and minimize the background (everything else).
  With the selection criteria defined we move on to scan the data in searching
  for a signal that matches our \textit{pdf} obtained from a Monte Carlo
  simulation based on the theoretical model for the dark matter candidate. If we
  can get a fit of the \textit{pdf} over data that is statistically significant
  and good (low enough chi square) we can claim to have found a new particle.

  This position is due to \href{10.1103/PhysRevD.106.012003}{my work} during
  PhD that searched for a dark matter, $Z'$
  candidate that couples to muons (a heavier electron). At TMU we are searching
  for a dark matter candidate that couples to taus (a heavier muon) or to
  neutrinos (no charge and no mass, invisible) an analysis that requires a much
  finer tuning to the selection criteria.

  \href{https://root.cern}{\texttt{CERN ROOT}} A \texttt{C/C++} framework
  for statistical analysis of high energy physics (HEP) data developed at CERN.
  I used the \textbf{standard fitter} which is based on \textbf{chi-square minimizer} (Minuit).
  Also a \textbf{Bayesian calculator} to estimate the confidence level of the coupling
  strength between $Z'$ and the other particles.

  % Analysis of all data taken in
  % the \href{https://belle.kek.jp}{Belle Experiment}, that has ran from 1999 to
  % 2010 acumulating 1000 $fb^{-1}$ of luminosity or $ ~ 10^9$ pairs of $B$ meson
  % particles, in search of alternative dark matter models. The Belle Detector was
  % not made to look specifically for dark matter candidates so in order to select
  % relevant data an elaborate system of selection criteria was needed.

  % The data obtained in the detector was stored in slices called skims that
  % contained a smaller portion of the data initially taken. In order to analyse
  % this data the Belle analysis framework (\texttt{basf}) was used. \texttt{basf}
  % is based in \texttt{C/C++} and with it one could select the charges, momentum,
  % mass, position and other variables related to each detected particle.

  % This set of variables could then be further refined to reconstruct a candidate
  % of dark matter particle that was produced by
  % the \href{https://belle.kek.jp/belle/transparency/accelerator1.html}{KEKB
  %   collider} and the particles that decayed from it detected by the Belle
  % Detector.

  % In order to know if the reconstructed particle we also simulate how a dark
  % matter particle would behave (signal) if it was produced by the KEKB collider and
  % detected by the Belle detector. We also simulate the main background for the dark
  % matter candidate. With a detailed study comparing the simulated signal and
  % background we can obtain a set of selection criteria that maximizes the figure
  % of merit (FOM).

  % With the selection criteria defined and tested we will look over the whole
  % Belle Experiment data, a bump search, and try to fit this bump. If
  % statistically significant we can claim to have found a new particle. If not we
  % can set limits on the theoretical parameters of the dark matter particles in
  % order to motivate new theories in a better informed direction.

\item[2018/12--2021/11] \textbf{Occupation:} Postdoctoral Researcher
  \\ \textbf{Institution:} Institute for Physics and Mathematics of the Universe
  (Kavli IPMU)
  \\ \textbf{Supervisor:} \href{https://db.ipmu.jp/member/personal/2440en.html}{Higuchi Takeo}
  \\ \textbf{Activities:} Statistical analysis and development of tools for the
  Time Dependent Charge Parity Violation (TDCPV) study group inside
  the \href{https://www.belle2.org/project/super_kekb_and_belle_ii}{Belle II
    experiment}. TDCPV can be defined as one of the ways to
  measure the lack of anti-matter in the universe. The Belle II experiment, an
  upgraded version of the Belle experiment which was designed to measure the
  Charge Parity Violation in the $B$ meson system, and it was responsible for
  \href{https://belle.kek.jp/belle/km_nobel/index.html}{the 2008 physics nobel
    prize}, has a better instrumentation than Belle and it will also collect 40
  times more data due to the upgraded collider Super KEKB.
  Initially I got acquainted with the Belle II analysis framework, \href{https://github.com/belle2/basf2}{\texttt{basf2}},
  similar to \texttt{basf} it is based in \texttt{C/C++} but it has an interface
  based on \texttt{python}.
  I reconstructed the $B^0 \rightarrow \psi(2S) K^{0}_{S}$ decay mode and
  evaluated how many valid events could be found in the short time Belle II operated.
  Later I worked on the reconstruction of the $B^0 \rightarrow D^* \ell
  \nu_\ell$ a decay mode that is unrelated to TDCPV but since it is one of the
  greatest portions of the neutral $B$ meson decay it is used as a control
  sample for the TDCPV analysis.
  The control sample study required a classification of the different background
  sources.

  This analysis was done using \texttt{ROOT} \texttt{python} bindings due to the
  use of the \texttt{basf2} \texttt{python} interface. Lists of variables could
  be imported from \texttt{basf2} into \texttt{ROOT} macros. Besides this, \texttt{pandas} \textbf{dataframes}
  were used to apply selection criteria upon the data as well as checking
  correlations and reducing size of original data samples. And finally
  \texttt{rootpandas} were used for file conversion from \texttt{.root} into
  \texttt{.ascii} since some of the original Belle tools for TDCPV analysis were
  still dependent on this format.

  Most of the Belle II experiment data taken and simulations are stored in a
  grid of computers around the institutions that participated in the Belle II
  collaboration so I also experienced some grid computing.

% \item[2016/12--2018/10] \textbf{Occupation:} English teacher
%   \\ \textbf{Institution:} Up-slope (\begin{CJK}{UTF8}{ipxm} NPO法人  アップ・スロープ  \end{CJK})
%   \\ \textbf{Activities:} English classes and conversation classes for children and
%   adults.

% \item[2016/08--2017/02] \textbf{Occupation:} Research Assistant
%   \\ \textbf{Institution:} Tohoku University Faculty of Science Department of
%   Physics (Particle Physics Experiment Group - Belle/Belle II/ ILC)
%   \\ \textbf{Supervisor:} \href{https://webific.ific.uv.es/web/en/content/prof-hitoshi-yamamoto-visiting-researcher-ific}{Yamamoto Hitoshi}
%   \\ \textbf{Activities:} Maintenance of the laboratory server's room.

% \item[2013/08--2014/08] \textbf{Occupation:} Teaching Assistant
% \\ \textbf{Institution:} Tohoku University Faculty of Science Department of
% Physics (Solid State Theory Group - Carbon Nanotubes)
% \\ \textbf{Supervisor:} \href{https://flex.phys.tohoku.ac.jp/~rsaito/rsaito-e.html}{Saito Riichiro}
% \\ \textbf{Activities:} Assisting exchange program students upon arrival and
% aiding them on their daily classes and research projects.

\item[2012/04--2013/04] \textbf{Occupation:} Data analyst
  \\ \textbf{Institution:} HSBC Asset Management Brazil
  \\ \textbf{Supervisor:} \href{https://br.linkedin.com/in/andrefontenele}{Fontenele André N.}
  \\ \textbf{Activities:} Development of a MySQL Database to manage fixed income
  investment funds.


\end{CV}

\section{Skills}

\begin{itemize}
	\item \textbf{Programming Languages}: \texttt{Python}, \texttt{C++}, \texttt{FORTRAN}, \LaTeX \\  and \textit{Mathematica}.
  \item \textbf{Statistics}: Parameter estimation, statistical tests, chi-square
    tests, fitting
  \item \textbf{Linux related}: \texttt{bash}, \texttt{zsh} shells, user
    management, network management and \texttt{docker}.
  \item \textbf{Machine Learning}: \href{https://root.cern/manual/tmva/}{\texttt{CERN ROOT} \texttt{TMVA}} package,
    multivariate analysis. Experience with Boosted Decision Trees developing non linear selection
    criteria with a set of discriminant variables to minimize background.
  \end{itemize}
\vspace{2\baselineskip}
\noindent Sagamihara, 2022/09/28


\end{document}

%Tabellen
\begin{table}[htbp] \centering%
\begin{tabular}{lll}\hline\hline
1 &amp; 2 &amp; 3 \\ \hline
1 &amp; \multicolumn{2}{c}{2} \\
\hline
\end{tabular}
\caption{Titel\label{Tabelle: Label}}
\end{table}