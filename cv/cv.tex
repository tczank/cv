\documentclass[a4paper]{article}
 
\usepackage{tabularx}
\usepackage [brazil, english, japanese]{babel}     % nomes e hifenaçã em português
\usepackage{CJKutf8}
\usepackage{graphicx} % Include figure files
\usepackage{t1enc}              % Permite digitar os acentos de forma normal
\usepackage[utf8]{inputenc}
\usepackage{ae}
\usepackage[T1]{fontenc}
\usepackage{CV}

\usepackage{hyperref}

\newcommand{\til}{\~{}}
\begin{document}
 
\pagestyle{empty}
 
%Ueberschrift
\begin{center}
\huge{\textsc{Curriculum Vitae}}
\vspace{\baselineskip}
 
\Large{\textsc{Czank Thomas Rafael}}
\end{center}
\vspace{1.5\baselineskip}

\section{Address}
\begin{CJK}{UTF8}{ipxm}
  \begin{flushleft}
〒252-0141  \\  
神奈川県相模原市緑区相原2ー8ー20 \\
ファミーユ相原102号室 \\
E-mail: thomas.czank@gmail.com\\
\end{flushleft}
\end{CJK}

\section{Personal Info}
\begin{minipage}{0.5\textwidth}
\begin{flushleft}
  Birth Date: 1989/12/29  \\
  Place of Birth: São Paulo-SP, Brazil \\
  Nationality: Brazilian \\
  Marital Status: Married\\
  Cellphone number: +81-080-3339-8149\\
\end{flushleft}
\end{minipage}
\hfill
\begin{minipage}{0.5\textwidth}
\begin{flushright}
\hspace*{-2cm}
\includegraphics[scale=2.]{/home/tczank/Dropbox/job_app/niigatadai/rireki/3x4_3.png}
\end{flushright}
\end{minipage}

\section{Education}
\begin{CV}
\item[2008/03--2012/12] Bachelor degree in Physics - University of São Paulo
\item[2013/04--2015/08] Master's degree in Physics - Tohoku University \\
  \hspace*{-2cm} (\url{Tip enhanced Raman spectroscopy of carbon nanotubes}{https://flex.phys.tohoku.ac.jp/japanese/riron/ronbun/m15thom.pdf})
\item[2015/09--2018/11] PhD degree in Physics - Tohoku University \\
  \hspace*{-3cm} (\url{The search for a dark sector muonic gauge boson $Z^'$ at the Belle experiment}{http://hdl.handle.net/10097/00125071})
\end{CV}



\section{Languages}
\begin{itemize}
	\item English - Fluent
	\item Spanish - Fluent
  \item Japanese (N2) - Advanced
  \item Portuguese - Native Speaker
\end{itemize}
 
%\section{Participação em Eventos}
%\begin{itemize}
%	\item Siicusp 2010-Simpósio Internacional de Iniciação Científica da Universidade de São Paulo
%\end{itemize}
 
\section{Professional Experience}
 
\begin{CV}
 
\item[2007/02--2018/11] \textbf{Occupation:} Private Tutor
\\ \textbf{Activities:} Tutoring on English, Math and Physics to High School
students preparing for University entrance examinations and also to University students.
%\begin{itemize}
%	\item \textbf{Preto} sempre;
%	\item \textbf{Colorido} quase nunca.
%\end{itemize}


%\item[02/2011--] \textbf{Ocupação:} Aluno de Iniciação Científica
%\\ \textbf{Instituição:} Laboratório de Instrumentação de Partículas no Departamento de Fìsica Geral no Instituto de Física da Universidade de São Paulo
%\\ \textbf{Professor:}Dr Marco Aurélio Lisboa Leite
%\\ \textbf{Atividades:} Acompanhamento e análise de eventos de calibração de Íons-Pesados do ATLAS e do ZDC(Zero Degree Calorimeter, Calorímetro de Zero Grau) detectores do Cern. E realizando simulações com o Fluka, um programa de física de altas energias com o método de Monte-Carlo, utilizando alvos de diferentes materiais e geometrias, e feixes de diferentes energias e partículas. 
 
\item[2012/04--2013/04] \textbf{Occupation:} Intern
\\ \textbf{Institution:} HSBC Asset Management Brazil
\\ \textbf{Supervisor:}Fontenele André N.
\\ \textbf{Activities:} Development of a mySQL Database to manage fixed income
investment funds.

\item[2013/08--2014/08] \textbf{Occupation:} Tutor
\\ \textbf{Institution:} Tohoku University Faculty of Science Department of
Physics (Solid State Theory Group - Carbon Nanotubes)
\\ \textbf{Supervisor:} Saito Riichiro
\\ \textbf{Activities:} Assisting exchange program students upon arrival and
aiding them on their daily classes and research projects.

\item[2016/08--2017/02] \textbf{Occupation:} Research Assistant
\\ \textbf{Institution:} Tohoku University Faculty of Science Department of
Physics (Particle Physics Experiment Group - Belle/Belle II/ ILC)
\\ \textbf{Supervisor:} Yamamoto Hitoshi
\\ \textbf{Activities:} Maintenance of the laboratory server's room.

\item[2016/12--2018/10] \textbf{Occupation:} English teacher
\\ \textbf{Institution:} Up-slope (\begin{CJK}{UTF8}{ipxm} NPO法人  アップ・スロープ  \end{CJK})
\\ \textbf{Activities:} English classes and conversation classes for children and
adults.

\item[2018/12--2021/11] \textbf{Occupation:} Project Researcher
\\ \textbf{Institution:} Institute for Physics and Mathematics of the Universe
(Kavli IPMU)
\\ \textbf{Supervisor:} Higuchi Takeo
\\ \textbf{Activities:} Working on the development of the analysis tools and
data analysis for the Time Dependent Charge Parity Violation (TDCPV) study group inside the Belle II experiment.

\item[2021/12--] \textbf{Occupation:} Project Researcher
  \\ \textbf{Institution:} Tokyo Metropolitan University
  \\ \textbf{Supervisor:} Kakuno Hidekazu
  \\ \textbf{Activities:} Working on dark sector alternate models analysis with
  Belle data
\end{CV}
 
\section{Skills}

\begin{itemize}
	\item \textbf{Operating Systems}; GNU/Linux e Microsoft Windows
	\item \textbf{Programming Languages}; Java, Python, C++, Fortran, \LaTeX, Mathematica, Matlab and mySql.
\end{itemize}
\vspace{2\baselineskip}
\noindent Sagamihara, 2022/09/26

 
\end{document}
 
%Tabellen
\begin{table}[htbp] \centering%
\begin{tabular}{lll}\hline\hline
1 &amp; 2 &amp; 3 \\ \hline
1 &amp; \multicolumn{2}{c}{2} \\
\hline
\end{tabular}
\caption{Titel\label{Tabelle: Label}}
\end{table}